\documentclass[presentation]{beamer}

\usetheme{Madrid}
\setbeamertemplate{navigation symbols}{}
\usecolortheme{beaver}
\usepackage{textpos}
\usepackage{amsmath}
\usepackage{graphicx}
\usepackage{color, colortbl}
\usepackage{multirow} 
\usepackage[scale=1.5]{ccicons}
\usepackage{caption}
%\usepackage[ddmmyyyy]{datetime}
\usepackage{ragged2e}
\usepackage{fancybox}
\usepackage{hyperref}
\hypersetup{
colorlinks=true,     
}
\usepackage[T1]{fontenc}
\usepackage{twcal}
\setbeamertemplate{caption}[numbered]
\setbeamertemplate{blocks}[rounded][shadow=true]

\definecolor{gold}{rgb}{0.85,.66,0}
 \definecolor{khaki}{rgb}{0.941176,0.901961,0.549020}
\definecolor{saffron}{rgb}{0.96, 0.77, 0.19}
\definecolor{sand}{rgb}{0.76, 0.7, 0.5}
\definecolor{ablue}{rgb}{0.94, 0.97, 1.0}
\definecolor{cgreen}{rgb}{0.0, 0.42, 0.24}
\setbeamercolor{bgcolor}{fg=black,bg=ablue}


\newenvironment<>{varblock}[2][.9\textwidth]{%
  \setlength{\textwidth}{#1}
  \begin{actionenv}#3%
    \def\insertblocktitle{#2}%
    \par%
    \usebeamertemplate{block begin}}
  {\par%
    \usebeamertemplate{block end}%
  \end{actionenv}}


\title[Workshop \LaTeX{}]{\textbf{Workshop on Document typesetting and Processing using \LaTeX}}
\author[\ccby{}  P. K. Yadav \& K. Kumar]{\vspace{-0.3cm}\textbf{Session: Tables in \LaTeX}}
\institute[SET/CE]{\href{http://www.texample.net/tikz/examples/rotated-polygons/}{
\includegraphics[scale=0.25]{fig/cover}}\centering\\
\textit{The rotated square}
\begin{flushleft}
Presented by: \textbf{P. K. Yadav \& K. Kumar} \\
\hspace{1.85cm}\textbf{Department of Civil Engineering} \hspace{2.5 cm} \today
\end{flushleft}
}
\date{\today}

\begin{document}

{
\setbeamertemplate{footline}{} 
\begin{frame}
\begin{textblock*}{50cm}(0\textwidth,0cm)
\includegraphics[width=4cm]{fig/logofinal}
\end{textblock*}
\vspace{1.2cm}


  \titlepage
  
  
\end{frame}
}
\addtocounter{framenumber}{-1}

%\logo{\includegraphics[width=1.5cm,height=1.5cm,keepaspectratio]{fig/logo}}



\addtobeamertemplate{frametitle}{}{%
\begin{textblock*}{500mm}(.93\textwidth,-0.95cm)
\includegraphics[width=0.9cm]{fig/mujlogo4}
\end{textblock*}}

\begin{frame}{Motivation}



\begin{enumerate}
\justifying

\item Tables are a common feature in academic writing, often used to summarise research results.\\
\vspace{0.3cm} 

\item Mastering the art of table construction in LaTeX is therefore necessary to produce quality document.\\
\vspace{0.3cm} 

\item LaTeX is not a spreadsheet, so designing a table may be time taking in the beginning.\\
\vspace{0.3cm} 

\item Customization of table can be done with several packages, e.g. \textcolor{red}{tabularx, tabu, colortbl, booktab}tabularx, and several others. 

\end{enumerate}






%\begin{block}{Block title}
%Hello
%\end{block}
%
%\begin{varblock}[7cm]{New block}
%  Variable width, here 4cm
%\end{varblock}

\end{frame}

\section{Table Environment}

\begin{frame}{The Table environment I}

Very similar to the figure environment.\\
\vspace{0.3cm}

\begin{columns}
\begin{column}{0.5\textwidth}

\begin{varblock}[7cm]{\textbf{The Table Environment}}

\textcolor{red}{$\backslash$begin\{table\}[Position]}\\
\textcolor{red}{$\backslash$begin\{tabular\}[Position]\{table spec\}}\\
\begin{center}
Table rows separated by \textcolor{red}{\textbf{\&}} ...
\end{center}
\textcolor{red}{$\backslash$caption\{table\}}
\textcolor{red}{$\backslash$end\{tabular\}[Position][pos]\{table spec\}}\\
\textcolor{red}{$\backslash$end\{table\}[Position]}

\end{varblock}
\end{column}
\end{columns}

\vspace{0.8cm}
%\begin{center}
%~
%    \begin{beamercolorbox}[rounded=true, center, shadow=true,wd=3cm]{bgcolor}
%      That is a good one
%    \end{beamercolorbox}
%~
%\end{center}

We will learn about all of these codes.

\end{frame}

\begin{frame}{The Table Environment II}

Next the \fcolorbox{gold}{ablue}{\textcolor{red}{$\backslash$begin\{table\}[Position]}}\\
\vspace{0.3cm}

The Position specifier in the table environment is exactly the same as was in the figure environment.\\

\vspace{0.3cm}

 \begin{columns}
 \begin{column}{0.5\textwidth}
 
 \begin{varblock}[7cm]{\textbf{Position specification}}
 
\begin{tabular}{|l|l|}
\hline 
\textcolor{red}{\textbf{h}} & where the table is declared (\textcolor{red}{h}ere)\\
\hline\hline
\textcolor{red}{\textbf{t}} & at the \textcolor{red}{t}op of the page\\
\hline\hline
\textcolor{red}{\textbf{b}} & at the \textcolor{red}{b}ottom of the page\\
\hline\hline
\textcolor{red}{\textbf{p}} & on the dedicate \textcolor{red}{p}age\\
\hline\hline
\textcolor{red}{\textbf{!}} & override the default float restrictions\\
\hline

\end{tabular}
 
 \end{varblock}
 \end{column}
 \end{columns}

\vspace{0.5cm}
The combination of position specifier, e.g. \textbf{htb}, can also be used.

\end{frame}


\begin{frame}{The Table Environment III}

We start with \fcolorbox{gold}{ablue}{\textcolor{red}{$\backslash$begin\{tabular\}[Position]\{table spec.\}}}\\
\vspace{0.2cm}

Here we have the \textbf{tabular} environment that requires \textbf{table specs} argument and an optional argument \textbf{position}.\\

\vspace{0.2cm}

The table spec. specifier identifies the number of column and the format of the separation between columns.
\\

\vspace{0.2cm}
 \begin{columns}
 \begin{column}{0.5\textwidth}
 
 \begin{varblock}[10.3cm]{\textbf{The table spec.}}
 
\begin{tabular}{|l|l|}
\hline 
\textcolor{red}{\textbf{l}} &  \textcolor{red}{l}eft-justified column\\
\hline\hline
\textcolor{red}{\textbf{c}} & \textcolor{red}{c}entered column \\
\hline\hline
\textcolor{red}{\textbf{r}} & \textcolor{red}{r}ight-justified column\\
\hline\hline
\textcolor{red}{\textbf{p\{`width'\}}} & specified column width\\
\hline\hline
\textcolor{red}{\textbf{|}} & vertical line separation between columns\\
\hline\hline
\textcolor{red}{\textbf{||}} & double vertical line separation between columns\\
\hline

\end{tabular}
 
 \end{varblock}
 \end{column}
 \end{columns}

\vspace{0.5cm}
The combination of position specifier, e.g. \textbf{htb}, can also be used.

\end{frame}

\begin{frame}{The Table Environment IV }

The \textbf{POSITION} specifier of the Tabular environment is only rarely used. We focus on the \fcolorbox{gold}{ablue}{\textcolor{red}{specifications required within the table body}}.\\
\vspace{0.2cm}

 \begin{columns}
 \begin{column}{0.5\textwidth}
 
 \begin{varblock}[11cm]{\textbf{Table row/columns specifier}}
\small
\begin{tabular}{|l|l|}
\hline 
\textcolor{red}{\textbf{\&}} &  column separator\\
\hline\hline
\textcolor{red}{\textbf{$\backslash\backslash$}} & start new row \\
\hline\hline
\textcolor{red}{\textbf{$\backslash$hline}} & horizontal line\\
\hline\hline
\textcolor{red}{\textbf{$\backslash$newline}} & start a new line within a cell \\
\hline\hline
\textcolor{red}{\textbf{$\backslash$cline\{i-j\}}} & partial horizontal line from column $i$ and  end of column $j$\\
\hline

\end{tabular}
 
 \end{varblock}
 \end{column}
 \end{columns}

\vspace{0.4cm}

We will use these specifiers in examples.

\end{frame}


\section{Examples}
\begin{frame}[fragile]{Examples of \LaTeX{} tables I }




\justifying

Lastly the \textbf{caption} code, which is exactly as it was with Figure Environment, only that \textcolor{red}{ \textbf{$\backslash$caption\{ Table Title\}}} is placed after the $\backslash$end\{ tabular\}. \\ 


 \begin{columns}
 \begin{column}{0.5\textwidth}

\begin{center}

    \begin{beamercolorbox}[rounded=true, center, shadow=true,wd=6cm]{bgcolor}
      \begin{verbatim}
     \begin{table}[b]
     %\centering
     \begin{tabular}{lcr}
      1 & 2 & 3 \\
       4 & 5 & 6 \\
       7 & 8 & 9 \\
     \end{tabular}
     \caption{The first table}
     \end{table}
           \end{verbatim}
    \end{beamercolorbox}

\end{center}
 \end{column}
 \hfill
  \begin{column}{0.3\textwidth}
  
  \begin{varblock}[3.5cm]{\textbf{A Simple Table}}
  
 \begin{table}[b]
 %\centering
 \begin{tabular}{lcr}
  1 & 2 & 3 \\
   4 & 5 & 6 \\
   7 & 8 & 9 \\
 \end{tabular}
 \caption{The first table}
  \vspace{-0.5cm}
 \end{table}
 
   \end{varblock}
  \end{column}

 \end{columns}




\end{frame}


\begin{frame}[fragile]{Examples of \LaTeX{} tables II}




\justifying

\textbf{Next we add vertical and horizontal lines to the table}. \\ 


 \begin{columns}
 \begin{column}{0.5\textwidth}

\begin{center}
\small
    \begin{beamercolorbox}[rounded=true, center, shadow=true,wd=6cm]{bgcolor}
      \begin{verbatim}
     \begin{table}[b]
     %\centering
     \begin{tabular}{|l||c|r|}
     \hline
      A & B & C \\
      \hline\hline
       1 & 2 & 3 \\
       \hline
       7 & 8 & 9 \\
       \hline
     \end{tabular}
     \caption{The first table}
     \end{table}
           \end{verbatim}
    \end{beamercolorbox}

\end{center}
 \end{column}
 \hfill
  \begin{column}{0.3\textwidth}
  
  \begin{varblock}[3.5cm]{\textbf{A Simple Table}}
  
 \begin{table}[b]
 %\centering
 \begin{tabular}{|l||c|r|}
 \hline
  A & B & C \\
   \hline \hline
   1 & 2 & 3 \\
    \hline
   7 & 8 & 9 \\
    \hline
 \end{tabular}
 \caption{A simple table}
 \end{table}
 \vspace{-0.5cm}
   \end{varblock}
  \end{column}

 \end{columns}

\end{frame}

\begin{frame}[fragile]{Examples of \LaTeX{} tables III}




\justifying
Next we use \verb|\cline{i-j}|. In order to create an empty row we simply \textcolor{red}{\&} to all columns.\\


 \begin{columns}
 \begin{column}{0.5\textwidth}

\begin{center}
\small
    \begin{beamercolorbox}[rounded=true, center, shadow=true,wd=6cm]{bgcolor}
      \begin{verbatim}
     \begin{table}[t]
     \begin{tabular}{|r|c|r|}
       \hline
       A & B & C\\
       \hline
       2 & 4 & Y\\ \cline{2-3}
        & 8 & N  \\
       \hline \hline
       8 &  7 & Y \\
       \hline
     \end{tabular}
     \caption{More table}
     \end{table}
           \end{verbatim}
           \vspace{-0.5cm}
    \end{beamercolorbox}

\end{center}
 \end{column}
 \hfill
  \begin{column}{0.3\textwidth}
  
  \begin{varblock}[3.5cm]{\textbf{More Table}}
  
\begin{table}[b]
\begin{tabular}{|r|c|r|}
  \hline
  A & B & C\\
  \hline
  2 & 4 & Y\\ \cline{2-3}
   & 8 & N  \\
  \hline \hline
  8 &  7 & Y \\
  \hline
\end{tabular}
\caption{More table}
\end{table}
 \vspace{-0.5cm}
   \end{varblock}
  \end{column}

 \end{columns}

\end{frame}


\begin{frame}[fragile]{Examples of \LaTeX{} tables IV}




\justifying
Multicolumn table can be created using specifier\\ \verb|\multicolumn{'num_cols'}{'alignment'}{'contents'}|\\
\verb|num_cols| is the number of subsequent columns to merge;\\ alignment is \verb| l, c, r, |, and \verb|content| is the actual data.\\


 \begin{columns}
 \begin{column}{0.5\textwidth}

\begin{center}
\scriptsize
    \begin{beamercolorbox}[rounded=true, center, shadow=true,wd=6cm]{bgcolor}
    \vspace{-0.5cm}
      \begin{verbatim}
    \begin{table}[h]
    \begin{tabular}{ |l|l| }
      \hline
      \multicolumn{2}{|c|}{Team sheet} \\
      \hline
      C  & M. Dhoni \\
      BM & V. Kohli \\
      BR & Z. Khan \\
      BR & I. Sharma \\
      \hline
    \end{tabular}
    \caption{A slightly complex table}
    \end{table}
           \end{verbatim}
           \vspace{-0.5cm}
    \end{beamercolorbox}

\end{center}
 \end{column}
 \hfill
  \begin{column}{0.3\textwidth}
  
  \begin{varblock}[3.5cm]{\textbf{More Table}}
  
\begin{table}[h]
    \begin{tabular}{ |l|l| }
      \hline
      \multicolumn{2}{|c|}{Team sheet} \\
      \hline
      C  & M. Dhoni \\
      BM & V. Kohli \\
      BR & Z. Khan \\
      BR & I. Sharma \\
      \hline
    \end{tabular}
    \caption{A slightly complex table}
    \end{table}
 \vspace{-0.5cm}
   \end{varblock}
  \end{column}

 \end{columns}

\end{frame}

\begin{frame}[fragile]{Examples of \LaTeX{} tables V}




\justifying
Colorful table can first by adding \verb|\usepackage[table]{xcolor}| in the preamble, and then using command: \\
\verb|\rowcolors{``starting row''}{``odd color''}{``even color''} |\\


 \begin{columns}
 \begin{column}{0.5\textwidth}

\begin{center}
    \begin{beamercolorbox}[rounded=true, center, shadow=true,wd=6cm]{bgcolor}
    \vspace{-0.5cm}
      \begin{verbatim}
    \begin{table}[h]
    \rowcolors{1}{green}{pink}
    \begin{tabular}{lll}
    odd     & odd   & odd \\
    even    & even  & even\\
    odd     & odd   & odd \\
    even    & even  & even\\
    \end{tabular}
    \caption{A colorful table}
    \end{table}
           \end{verbatim}
           \vspace{-0.5cm}
    \end{beamercolorbox}

\end{center}
 \end{column}
 \hfill
  \begin{column}{0.3\textwidth}
  
  \begin{varblock}[3.5cm]{\textbf{More Table}}
  
\begin{table}[h]
\begin{tabular}{lll}
\rowcolor{green}
odd     & odd   & odd \\
\rowcolor{pink}
even    & even  & even\\
odd     & odd   & odd \\
\rowcolor{green}
even    & even  & even\\
\end{tabular}
\caption{A colorful table}
\end{table}
 \vspace{-0.5cm}
   \end{varblock}
  \end{column}

 \end{columns}

\end{frame}

\section{Conclusion}

\begin{frame}{Conclusion}
\begin{itemize}

\item Tables in \LaTeX{} requires practice.
\vspace{0.3cm}
\item There exist several free software that let you convert spreadsheet to \LaTeX{} table, e.g. excel2latex, matrix2latex. Details can be found here: \href{http://en.wikibooks.org/wiki/LaTeX/Tables}{Using Spreadsheet}.
\vspace{0.3cm}
\item There are several packages that facilitates the design of complex tables, e.g. tables spanning several pages, footnotes in table, margin formatting. More details can be found \href{http://en.wikibooks.org/wiki/LaTeX/Tables}{here} and \href{http://www.tug.org/tutorials/tugindia/}{here}.
\vspace{0.3cm}

\item A handy tool that can be used to make tables can be found \href{http://authortools.aas.org/LATEX/make-latex.html}{here}, \href{http://www.ctan.org/pkg/latable}{here} and \href{http://truben.no/latex/table/}{here (online)}. 


\end{itemize}
\end{frame}

{
\setbeamertemplate{footline}{} 
\begin{frame}{ \textbf{Next $\ldots$,}}
\pagestyle{plain}
\Huge

\textbf{Equations, Bibliography... further improve our \LaTeX{} document.}


\end{frame}

}


\end{document}