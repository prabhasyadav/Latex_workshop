\documentclass[presentation]{beamer}

\usetheme{Goettingen}
\setbeamertemplate{navigation symbols}{}
%\usecolortheme{beaver}
\usepackage{textpos}
\usepackage{amsmath}
\usepackage{graphicx}
\usepackage{color, colortbl}
\usepackage{multirow} 
\usepackage{verbatimbox}
\usepackage{caption}
\usepackage[scale=1]{ccicons}
\usepackage{empheq}
%\usepackage[ddmmyyyy]{datetime}
\usepackage{ragged2e}
\usepackage{fancybox}
\usepackage{hyperref}
\hypersetup{
colorlinks=true,     
}
\usepackage[T1]{fontenc}
\usepackage{twcal}
\setbeamertemplate{caption}[numbered]
\setbeamertemplate{blocks}[rounded][shadow=true]
\setbeamerfont{page number in head/foot}{size=\large}
%\setbeamertemplate{footline}[frame number]
\setbeamertemplate{footline}[text line]{%
 \parbox{\linewidth}{\vspace*{-14pt} \includegraphics[width=1.0cm,height=1.2cm,keepaspectratio]{fig/mujlogo4}} \hspace{1.5cm}\insertframenumber/\inserttotalframenumber}


\newcommand*\widefbox[1]{\fbox{\hspace{2em}#1\hspace{2em}}}

\definecolor{gold}{rgb}{0.85,.66,0}
 \definecolor{khaki}{rgb}{0.941176,0.901961,0.549020}
\definecolor{saffron}{rgb}{0.96, 0.77, 0.19}
\definecolor{sand}{rgb}{0.76, 0.7, 0.5}
\definecolor{ablue}{rgb}{0.94, 0.97, 1.0}
\definecolor{cgreen}{rgb}{0.0, 0.42, 0.24}
\setbeamercolor{bgcolor}{fg=black,bg=ablue}
\setbeamercolor{block title}{fg=white,bg=orange!20!black}
\setbeamercolor{block body}{fg=black,bg=ablue}

\newenvironment<>{varblock}[2][.9\textwidth]{%
  \setlength{\textwidth}{#1}
  \begin{actionenv}#3%
    \def\insertblocktitle{#2}%
    \par%
    \usebeamertemplate{block begin}}
  {\par%
    \usebeamertemplate{block end}%
  \end{actionenv}}


\title[Workshop \LaTeX{}]{\textbf{Workshop on Document typesetting and Processing using \LaTeX}}
\author[\ccby{}  P. K. Yadav \& K. Kumar
 ]{\vspace{-0.3cm}\textbf{Session: Equations in \LaTeX}}
\institute[SET/CE]{\href{http://www.texample.net/tikz/examples/unit-circle/}{
\includegraphics[scale=0.18]{fig/cover}}\centering\\
\textit{A unit circle}
\begin{flushleft}
Presented by: \textbf{P. K. Yadav \& K. Kumar} \\
\hspace{1.85cm}\textbf{Department of Civil Engineering} \hspace{1 cm} \today
\end{flushleft}
}
\date{\today}

\begin{document}

{
\setbeamertemplate{footline}{} 
\begin{frame}
\begin{textblock*}{50cm}(0\textwidth,0cm)
\includegraphics[width=5cm]{fig/logofinal}
\end{textblock*}
\vspace{1.2cm}


  \titlepage
  
  
\end{frame}
}
\addtocounter{framenumber}{-1}




\section{Motivation}
\begin{frame}[fragile]{Motivation}




\justifying
Let us start with a quote from the \TeX{} creator.

\begin{quote}
``\TeX{} is a
new typesetting system intended for the creation of beautiful books- and especially for books
that contain a lot of mathematics''
\end{quote}


\textbf{In-line mathematics}/chemical equations are very poorly printed from word processors.\\
\vspace{0.2cm}


The \textbf{numbering and cross-referencing} of mathematical equation still remains complicated in word processors. \\
\vspace{0.2cm}



\LaTeX{} solves these. An example:\\
\vspace{0.2cm}



%\scriptsize{
%\begin{verbatim}
%Using (5.64) and the fact that the
%$c_n=\langle\psi_n\vert\Psi\rangle$
%and $d_n^*=\langle X\psi_n\rangle$,
%the scalar product $\langle X\vert
%\Psi\rangle$ can be expressed in the
%way as $\langle X\vert\Psi\rangle=
%\sum_nd_n^*c_n = \mathbf{d}^\dagger
%\boldsymbol{\cdot}\mathbf{c}$ where
%\(\mathbf{c}\) is a column vector
%with elements $c_n$ and row vector
%$\mathbf{d}^\dagger$ with elements
%$d_n^*$.
%\end{verbatim}
%}


\centering
\fcolorbox{red}{ablue}{\parbox{8cm}{

Using (5.64) and the fact that the
$c_n=\langle\psi_n\vert\Psi\rangle$
and $d_nˆ*=\langle X\psi_n\rangle$,
the scalar product $\langle X\vert
\Psi\rangle$ can be expressed in the
way as $\langle X\vert\Psi\rangle=
\sum_nd_nˆ*c_n = \mathbf{d}ˆ\dagger
\boldsymbol{\cdot}\mathbf{c}$ where
\(\mathbf{c}\) is a column vector
with elements $c_n$ and row vector
$\mathbf{d}ˆ\dagger$ with elements
$d_nˆ*$.
}}



\end{frame}

\section{Equation Environment}
\begin{frame}{\textbf{The equation environment I}}

A mathematical texts contains in-line equation and separate equations. These must be first identified.\\
\vspace{0.6 cm}


Text (e.g. $\sin$ ) and mathematical symbol (e.g. $\beta$) are typed differently (i.e. normal and italic , $\sin\beta$).\\
\vspace{0.6 cm}


The spacing between texts in mathematical equation are
different from texts in paragraph.\\
\vspace{0.6 cm}


We identify these first.

\end{frame}


\begin{frame}[fragile]{\textbf{The equation environment II}}



\justifying
Mathematics environments:\\
\vspace{0.2cm}


For \textbf{inline math} typesetting we use: \textcolor{red}{\$...\$}.\\
\vspace{0.2cm}


e.g. \verb|$x^2+y^2= a^2 \times \sin\theta$| will print \\
\vspace{0.2cm}

\begin{center}
 \fbox{$x^2+y^2=a^2\times \sin\theta$}
\end{center}



For the equations, we use the \textbf{equation} environment, i.e. \textcolor{red}{$\backslash$begin\{equation\}...$\backslash$end\{equation\}}. e.g.\\


 \begin{columns}
 \begin{column}{0.5\textwidth}

\begin{center}
\small
    \begin{beamercolorbox}[rounded=true, center, shadow=true,wd=4.5cm]{bgcolor}
    \vspace{-0.5cm}
      \begin{verbatim}
    \begin{equation}
   x^2+y^2 = a^2 \times 
   \sin\theta
    \end{equation}
           \end{verbatim}
           \vspace{-0.5cm}
    \end{beamercolorbox}

\end{center}
 \end{column}
 \hfill
  \begin{column}{0.3\textwidth}
  
  \begin{beamercolorbox}[rounded=true, center, shadow=true,wd=4cm]{bgcolor}
      \vspace{-0.5cm}
      \begin{equation}
        x^2+y^2 = a^2 \times \sin\theta
         \end{equation}
 \vspace{-0.7cm}
   \end{beamercolorbox}
  \end{column}

 \end{columns}

\vspace{0.2cm}
Try: \textcolor{red}{$\backslash$begin\{equation$^\ast$\}...$\backslash$end\{equation$^\ast$\}}, what results.\\

\end{frame}



\begin{frame}[fragile]{\textbf{The equation environment III}} {Texts and spaces in maths}



\justifying
Mathematics environments:\\
\vspace{0.2cm}
Text within the inline maths or eqaution environment can be inserted using \textcolor{red}{$\backslash$textrm\{text\}} specifier. e.g.,

 \begin{columns}
 \begin{column}{0.4\textwidth}

\begin{center}
\small
    \begin{beamercolorbox}[rounded=true, center, shadow=true,wd=4cm]{bgcolor}
    \vspace{-0.5cm}
      \begin{verbatim}
    \begin{equation}
   x^2+y^2 = a^2  
   \textrm{for} 
   a = \sin\theta
    \end{equation}
           \end{verbatim}
           \vspace{-0.5cm}
    \end{beamercolorbox}

\end{center}
 \end{column}
 \hfill
  \begin{column}{0.3\textwidth}
  \small
  
  \begin{beamercolorbox}[rounded=true, center, shadow=true,wd=5cm]{bgcolor}
      \vspace{-0.5cm}
      \begin{equation}
        x^2+y^2 = a^2 \textrm{for} a= \sin\theta
         \end{equation}
 \vspace{-0.5cm}
   \end{beamercolorbox}
  \end{column}

 \end{columns}
 
 \vspace{0.3cm}

Eq (2) is not perfect because we need space between $a^2$, for and $a$. Spaces in the math environment is specified with:






%\fcolorbox{red}{white}{\parbox{8cm}{Using (5.64) and the fact that the
%$c_n=\langle\psi_n\vert\Psi\rangle$
%and $d_nˆ*=\langle X\psi_n\rangle$,
%the scalar product $\langle X\vert
%\Psi\rangle$ can be expressed in the}}



\end{frame}


\begin{frame}[fragile]{\textbf{The equation environment IV}}



\textbf{Spaces} in the math environment is specified with:

\begin{columns}
 \begin{column}{0.5\textwidth}
 
 \begin{varblock}[3.8cm]{\textbf{Space specification}}
 
\begin{tabular}{|l|l|}
\hline 
\textcolor{red}{\textbf{$\backslash$,}} & small space\\
\hline\hline
\textcolor{red}{\textbf{$\backslash$:}} & medium space\\
\hline\hline
\textcolor{red}{\textbf{$\backslash$\;}} & large space\\
\hline\hline
\textcolor{red}{\textbf{$\backslash$!}} & negative space\\
\hline

\end{tabular}
 
 \end{varblock}
 \end{column}
 \end{columns}

\vspace{0.3cm}
Our examples can be modified as:

 \begin{columns}
 \begin{column}{0.4\textwidth}

\begin{center}
\small
    \begin{beamercolorbox}[rounded=true, center, shadow=true,wd=4cm]{bgcolor}
    \vspace{-0.5cm}
      \begin{verbatim}
    \begin{equation}
   x^2+y^2 = a^2\;  
   \textrm{for} \; 
   a = \sin\theta
    \end{equation}
           \end{verbatim}
           \vspace{-0.5cm}
    \end{beamercolorbox}

\end{center}
 \end{column}
 \hfill
  \begin{column}{0.3\textwidth}
  \small
  
  \begin{beamercolorbox}[rounded=true, center, shadow=true,wd=5cm]{bgcolor}
      \vspace{-0.5cm}
      \begin{equation}
        x^2+y^2 = a^2\; \textrm{for}\; a= \sin\theta
         \end{equation}
 \vspace{-0.4cm}
   \end{beamercolorbox}
  \end{column}

 \end{columns}
 
 \vspace{0.3cm}

\end{frame}

\section{Maths Typsetting Examples}

\begin{frame}[fragile]{\textbf{Examples: Equation without numbers}}


 \begin{beamercolorbox}[rounded=true, center, shadow=true,wd=10cm]{bgcolor}
    \vspace{-0.5cm}
      \begin{verbatim}
\begin{equation*}
    \left(\int_{-\infty}^{\infty} e^{-x^2}\right)=
    \int_{-\infty}^{\infty}\int_{-\infty}^{\infty}
    e^{-(x^2+y^2)}dx\,dy
\end{equation*}
           \end{verbatim}
           \vspace{-0.5cm}
    \end{beamercolorbox}

\vspace{0.3cm}
results to: 

\begin{empheq}[box=\widefbox]{equation*}
    \left(\int_{-\infty}^{\infty} e^{-x^2}\right)=
    \int_{-\infty}^{\infty}\int_{-\infty}^{\infty}
    e^{-(x^2+y^2)}dx\,dy
\end{empheq}

The $\ast$ after equation avoids numbering of equation. The $\backslash$left( ... $\backslash$right) produces the perfect brackets.

\end{frame}

\begin{frame}[fragile]{\textbf{Examples II: Framed and Roots}}
\textbf{\textcolor{red}{Framed}}\\
\vspace{0.2cm}

\small{
 \begin{beamercolorbox}[rounded=true, center, shadow=true,wd=9cm]{bgcolor}
    \vspace{-0.5cm}
      \begin{verbatim}
\begin{equation}
\boxed{\int_0^\infty f(x)\,{\textrm{d}}x
\approx\sum_{i=1}^nw_i{\textrm{e} }^{x_i}f(x_i)}
\end{equation}
           \end{verbatim}
           \vspace{-0.3cm}
    \end{beamercolorbox}
}
results to: 
 \vspace{-0.2cm}
\begin{equation*}
\boxed{\int_0^\infty f(x)\,{\textrm{d}}x
\approx\sum_{i=1}^nw_i{\textrm{e} }^{x_i}f(x_i)}
\end{equation*}

\textbf{\textcolor{red}{Roots}}\\
\vspace{0.1cm}

\small{
 \begin{beamercolorbox}[rounded=true, center, shadow=true,wd=7.5cm]{bgcolor}
    \vspace{-0.5cm}
      \begin{verbatim}
\begin{equation}
\sqrt[n]{\frac{x^n-y^n}{1+u^{2n}}}
\end{equation}
           \end{verbatim}
           \vspace{-0.5cm}
    \end{beamercolorbox}
  }  
 
    results to: 
 \vspace{-0.1cm}
\begin{equation*}
\boxed{\sqrt[n]{\frac{x^n-y^n}{1+u^{2n}}}}
\end{equation*}



\end{frame}


\begin{frame}[fragile]{\textbf{Examples III: Array and Cases}}

\textbf{\textcolor{red}{Array}}


{\scriptsize
 \begin{beamercolorbox}[rounded=true, center, shadow=true,wd=9cm]{bgcolor}
    \vspace{-0.5cm}
    \begin{verbatim}
    \begin{equation}
      \begin{array}{lcll}
\psi(x,t) &=& A({\textrm{e}^{\textrm{i}kx}-{\textrm{e}^
{\textrm{-i}kx}e^{\textrm{-i}\omega t}&\\
&=& D\sin kxe^{\textrm{-i}\omega t}, & D = 2\textrm{i}A
           \end{array}
            \end{equation}
            \end{verbatim}
           \vspace{-0.5cm}
    \end{beamercolorbox}
}
 \vspace{0.1cm}
    results to: 
 \vspace{-0.2cm}
\begin{equation*}
 \begin{array}{lcll}
\psi(x,t) &=& A({\textrm{e}^{\textrm{i}kx}}-{\textrm{e}^
{\textrm{-i}kx}}e^{\textrm{-i}\omega t}&\\
&=& D\sin kxe^{\textrm{-i}\omega t}, & D = 2\textrm{i}A
           \end{array}
\end{equation*}


\textbf{\textcolor{red}{Cases, using ``amsmath'' package}}


{\scriptsize
 \begin{beamercolorbox}[rounded=true, center, shadow=true,wd=9cm]{bgcolor}
    \vspace{-0.5cm}
    \begin{verbatim}
      \begin{equation}
f(n) = \begin{cases} n/2 &\textrm{if } n = 0 \\ 
(3n +1)/2 & \textrm{if } n \neq 1. \end{cases} 
           \end{equation}
            \end{verbatim}
           \vspace{-0.5cm}
    \end{beamercolorbox}
}
 \vspace{0.1cm}
    results to: 
 \vspace{-0.2cm}
\begin{equation*}
 f(n) = \begin{cases} n/2 &\textrm{if } n = 0 \\ 
 (3n +1)/2 & \textrm{if } n \neq 1. \end{cases} 
\end{equation*}

\end{frame}


\begin{frame}[fragile]{\textbf{Examples IV: Align and Substack}}


\textbf{\textcolor{red}{Align}}, using ``amsmath'' package, environment is used for two or more equations when vertical alignment is desired.

{\small
 \begin{beamercolorbox}[rounded=true, center, shadow=true,wd=8cm]{bgcolor}
    \vspace{-0.5cm}
    \begin{verbatim}
   \begin{align*}
   u &= \arctan x & dv &= 1 \, dx
   \\ du &= \frac{1}{1 + x^2} dx & v &= x.
   \end{align*}
            \end{verbatim}
           \vspace{-0.5cm}
    \end{beamercolorbox}

 \vspace{0.1cm}
    results to: 
 \vspace{-0.4cm}
\begin{align*}
u &= \arctan x & dv &= 1 \, dx
\\ du &= \frac{1}{1 + x^2} dx & v &= x.
\end{align*}
}
   \vspace{-0.5cm}
\textbf{\textcolor{red}{substack}}

{\small
 \begin{beamercolorbox}[rounded=true, center, shadow=true,wd=8cm]{bgcolor}
    \vspace{-0.5cm}
    \begin{verbatim}
  \begin{equation*}
  \sum_{\substack{0\leq i\leq m\\ 0jn}}
  \end{equation*}
            \end{verbatim}
           \vspace{-0.5cm}
    \end{beamercolorbox}

 \vspace{0.1cm}
    results to: 
 \vspace{-0.4cm}
\begin{equation*}
\sum_{\substack{0\leq i\leq m\\ 0<j<n}}
\end{equation*}
 \vspace{-0.2cm}
\begin{flushright}
\textbf{How about substack on the top of sum symbol?}
\end{flushright}
}
\end{frame}


\begin{frame}[fragile]{\textbf{Examples V: Split Environment}}



\textbf{\textcolor{red}{Split}}, using ``amsmath'' package, is for single equations that are too long to fit on a single line and hence
must be split into multiple lines. 


 \begin{beamercolorbox}[rounded=true, center, shadow=true,wd=8cm]{bgcolor}
    \vspace{-0.2cm}
    \begin{verbatim}
   \begin{equation}
   \begin{split}
   (x+y+z)^2 & = x^2+xy+xz \\
   & + xy + y^2 + yz \\
   & + xz + yz + z^2
   \end{split}
   \end{equation}
            \end{verbatim}
           \vspace{-0.5cm}
    \end{beamercolorbox}

 \vspace{0.1cm}
    results to:   
 \fbox{
 \parbox{5cm}{
   \vspace{-0.4cm}
 \begin{equation*}
  \begin{split} 
   (x+y+z)^2 & = x^2+xy+xz \\
   & + xy + y^2 + yz \\
   & + xz + yz + z^2
   \end{split}
   \end{equation*}
      \vspace{-0.4cm}
}}

\end{frame}

\begin{frame}[fragile]{\textbf{Examples VI: Matrices environment}}
Matrix can be represented in many ways. amsmath package provide several options:
\small
 \begin{beamercolorbox}[rounded=true, center, shadow=true,wd=9.5cm]{bgcolor}
    \vspace{-0.3cm}
    \begin{verbatim}
  \begin{gather*}
  \begin{matrix} 0 & 1\\ 1 & 0 \end{matrix}\qquad
  \begin{pmatrix} 0 & -i\\ i & 0 \end{pmatrix}\qquad
  \begin{bmatrix} a & b\\ c & d \end{bmatrix}\qquad
  \begin{vmatrix} 0 & 1\\ -1 & 0 \end{vmatrix}\qquad
  \begin{Vmatrix} f & g\\ e & v \end{Vmatrix}
  \end{gather*}
            \end{verbatim}
           \vspace{-0.5cm}
    \end{beamercolorbox}

 \vspace{0.1cm}
    results to:   \\
     \vspace{0.3cm}
     
     
 \fbox{
 \parbox{9cm}{
   \vspace{-0.4cm}
   \begin{gather*}
   \begin{matrix} 0 & 1\\ 1 & 0 \end{matrix}\qquad
   \begin{pmatrix} 0 & -i\\ i & 0 \end{pmatrix}\qquad
   \begin{bmatrix} a & b\\ c & d \end{bmatrix}\qquad
   \begin{vmatrix} 0 & 1\\ -1 & 0 \end{vmatrix}\qquad
   \begin{Vmatrix} f & g\\ e & v \end{Vmatrix}
   \end{gather*}
      \vspace{-0.4cm}
}}

   \vspace{0.3cm}
     

The \textbf{gather environment} gathers and centers equations.\textbf{$\mathbf{\backslash}$qquad} is used for large spacing. 

\end{frame}


\section{End Remarks}

\begin{frame}{End Remarks}

\justifying

Math mode in \LaTeX{} is quite like an art than writing a mere equation. For advancement one may consider: 
\vspace{0.3cm}

\begin{enumerate}

\item \href{http://en.wikibooks.org/wiki/LaTeX/Advanced_Mathematics}{Wikipedia} provides a quite comprehensive detail on advanced mathematics typesetting.
\vspace{0.3cm}

\item An exhaustive list of mathematical symbol in \LaTeX{}, can be found \href{http://www.ctan.org/tex-archive/info/symbols/comprehensive}{here}, and tool to find the proper symbol can be found \href{http://detexify.kirelabs.org/classify.html}{here (online tool)}
\vspace{0.3cm}

\item Tools for math typesetting can be found \href{http://latexeqedit.sourceforge.net/index.php}{here} and \href{http://www.sciweavers.org/free-online-latex-equation-editor}{online tool here}.

\item A very nice \LaTeX{} review for practice can be found \href{http://www.sharelatex.com/learn/Main_Page}{here}.
\end{enumerate}




\end{frame}



{
\setbeamertemplate{footline}{} 
\begin{frame}{ \textbf{Next $\ldots$,}}
\pagestyle{plain}
\Huge

\textbf{Bibliography, Cross-referencing... \\ \vspace{0.5cm}Further improve our \LaTeX{} document.}


\end{frame}





}


\end{document}