%This is preamble
\documentclass[11pt, a4paper]{article}
\usepackage{amsmath}
\usepackage{amstext}
\usepackage{amssymb}
\usepackage{graphicx}
\usepackage[table]{xcolor}
\usepackage [square]{natbib} % options: round, square, comma, authoryear, numbers
\bibliographystyle{IEEEtranN} % plain, rsc, abbrvnat etc
%\usepackage{lineno}

%\usepackage{hyperref}
%\pagestyle{headings}

% These are front-matters
\begin{document}

\title{My First \LaTeX{} document}
\author{P. K. Yadav$^1$\thanks{Corresponding Author: P.K. Yadav, \newline rs@pm.0.in},  K. Kumar$^2$, F. A. Faroque$^3$ \\\\ $^1$Manipal University Jaipur, $^2$PMO Office, Delhi, $3$CM Office Srinagar, J \& K}


\maketitle

%\linenumbers
\begin{abstract}\label{abs}
Why waste time check results in section (\ref{res})
\end{abstract}


% This is where the main body starts
\section{Introduction}\label{Intro}

Let us start with the list. I am learning \LaTeX{} because of the following:
%\renewcommand\theenumi{\Roman{enumi}}
%\renewcommand\labelenumi{[{\bfseries\theenumi}]}
\begin{enumerate}
\item I find it cool
\item I find it hot
\item I can not sense temperature
\end{enumerate}

A sum of square of $x$ and that of $y$ produces a square of $a$, mathematically, $x^2+y^2 = a^2$. We may represent it as:

\begin{equation}
f(x,y) = a^2
\end{equation}


\subsection{Actual Problem}\label{ActProb}
The nested Lists.
\renewcommand\labelitemi{$\square$}
\renewcommand{\labelitemii}{$\diamond$}
\begin{itemize}
\item The {\tt itemize} label at the first level is a bullet.
\begin{enumerate}
\item The numbering is with Arabic numerals since this is ...
\begin{itemize}
\item This is the third level of the nesting, but the ...
\begin{enumerate}
\item And this is the fourth level of the overall ...
\item Thus the numbering is with lower case letters ...
\end{enumerate}
\item The label at this level is a long dash.
\end{itemize}
\item Every list should contain at least two points.
\end{enumerate}
\item Blank lines ahead of an ...
\end{itemize}

\section{Results}\label{res}

Let us insert a figure here. A standard scientific document has figure at top or at the bottom of the page.

\begin{figure}[t]
\includegraphics[scale=0.4, angle=123.7]{fig/logo} %\centering
\caption{The first figure, Sharda Uni Logo.}
\end{figure}


\section{Discussions}

\begin{figure}[b]
\includegraphics[scale=0.2, width = 5cm, angle=270]{fig/logo} %\centering
\caption{The first figure, Sharda Uni Logo.}
\end{figure}

\subsection{\LaTeX{} tables}
The simplest table.

\begin{table}[t]
%\centering
\begin{tabular}{lcr}
 1 & 2 & 3 \\
  4 & 5 & 6 \\
  7 & 8 & 9 \\
\end{tabular}
\caption{The first table}
\end{table}

\subsubsection{Case-Study I}

Table with \verb|\cline{i-j}|
\begin{table}[t]
\begin{tabular}{|r|c|r|}
  \hline
  A & B & C\\
  \hline
  2 & 4 & Y\\ \cline{2-3}
   & 8 & N  \\
  \hline \hline
  8 &  7 & Y \\
  \hline
\end{tabular}
\caption{More table}
\end{table}
\subsubsection{A slightly complex table}

Merging several columns, we follow \cite{lamport94}.
\begin{table}[t]
\begin{tabular}{ |l|l| }
  \hline
  \multicolumn{2}{|c|}{Team sheet} \\
  \hline
  C  & M. Dhoni \\
  BM & V. Kohli \\
  BM & R. Sharma \\
  BR & Z. Khan \\
  BR & I. Sharma \\
  \hline
\end{tabular}
\caption{A slightly complex table}
\end{table}
\subsubsection{Colorful table}

\begin{center}
\begin{table}[h]
\rowcolors{1}{green}{pink}
\begin{tabular}{lll}
odd     & odd   & odd \\
even    & even  & even\\
odd     & odd   & odd \\
even    & even  & even\\
\end{tabular}
\caption{A colorful table}
\end{table}
\end{center}


\section{Conclusions}
\subsection{Future works}



% These are end-matters

\section{Appendices}
\section{Acknowledgements} %* makes puts off the section number
We appreciate the great work of \cite{lamport94}, which let us write our work more beautifully. Of course we also thank also Prof. Knuth for his grand work \cite[p.~215]{knuth2014}.


%\begin{thebibliography}{99}
%\bibitem{lamport94}
%       Leslie Lamport,
%       \emph{\LaTeX{}: A Document Preparation System}.
%       Addison Wesley, Massachusetts,
%       2nd Edition,
%       1994.
%       
%\bibitem{knuth2014} 
%\emph{nuth, Donald Ervin. ``The Art of Computer Programming (TAOCP)"}. Retrieved February 01, 2014.
%\end{thebibliography}

\begin{thebibliography}{10}
\bibitem[Leslie(1985)]{lamport94} Leslie Lamport, 1985. \LaTeX{}-A Document Preparation...
\bibitem[Donale(2000)]{knuth2014}Donald E. Knuth, 1989. Typesetting Concrete Mathematics,...
\bibitem[Ronald, Donald and Ore(1989)]{rondon89}Ronald L. Graham, ...
\end{thebibliography}


\end{document}