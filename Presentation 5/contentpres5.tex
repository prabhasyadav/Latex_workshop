% DO NOT COMPILE THIS FILE DIRECTLY!
% This is included by the other .tex files.

\begin{frame}[t,plain]
\titlepage
\end{frame}

\section{Motivation}
\begin{frame}[t]{\textbf{Motivation}}
\vspace{0.2cm}
\justifying

Standard documents require proper referencing and bibliography. Key points with \LaTeX{} bibliography management are:
\vspace{0.2cm}
\begin{enumerate}

\item Very flexible bibliography styling.\\
\vspace{0.2cm}
\item Very simple conversion between different bibliography style.\\
\vspace{0.2cm}
\item Flexible cross-referencing and hyper-linking.\\
\vspace{0.2cm}
\item Possibility to manage references material using independent tools e.g. Endnote$^\circledR$, Bibtex.\\
\vspace{0.2cm}

We will explore first a basic referencing in \LaTeX{} and follow it up with a very flexible NATBIB package. 
\end{enumerate}

\end{frame}

\section{thebibliography Environment}
\begin{frame}[t,fragile]{\textbf{thebibliography Environment I}}


\justifying
Referencing involves two information: The \textbf{Citation} and the detail of the citation- the \textbf{complete reference information}.\\ We with start with the latter.
\vspace{0.1cm}

The environment for providing reference information is: \textcolor{red}{$\backslash$begin\{ thebibliography\}\{options\}}. A typical code is:\\
\vspace{0.2cm}
\small
\begin{beamercolorbox}[ rounded=true, center, shadow=true,wd=9.5cm]{bgcolor}
      \vspace{-0.5cm}
\begin{verbatim}
     \begin{thebibliography}{9}
   
     \bibitem{lamport94}
       Leslie Lamport,
       \emph{\LaTeX: A Document Preparation System}.
       Addison Wesley, Massachusetts,
       2nd Edition,
       1994.
     \end{thebibliography}
 \end{verbatim}
 \vspace{-0.7cm}
   \end{beamercolorbox}


\end{frame}

\begin{frame}[t,fragile]{\textbf{thebibliography Environment II}}

\justifying


The \textbf{options} in \textcolor{red}{$\backslash$begin\{ thebibliography\}\{options\}} refers to the width of the widest label in the bibliography. We use the value $\mathbf{1-9}$ if the document has up to \textbf{9} references, and\textbf{ any two digit number}, e.g. 56, if the document has \textbf{9 and 99} references.\\
\vspace{0.3cm}

In \textcolor{red}{$\backslash$bibitem\{lamport94\}}, $\backslash$bibitem specifier which identifies \textbf{``lamport94''} as the citation. The identifier can be in any format,\\
\vspace{0.3cm}

Finally, we use \textcolor{red}{$\backslash$cite[options]\{lamport94\}} to cite in the text. In \textbf{options} we can add any text information that we want to appear in the citation, e.g. [p. 215], referring to page number 215 of the Lamport book published in 94.

\end{frame}

\begin{frame}[t,fragile]{\textbf{thebibliography Environment III}}


Lets us look at an example.

\scriptsize
\begin{varblock}[9cm]{\textbf{Refernce Citing Example}}
\vspace{-0.3cm}
\begin{verbatim}
We appreciate the great work in \cite{knuth2014}, 
which let us write our work more beautifully
\cite[p. 200]{lamport94}. 
\end{verbatim}
\vspace{-0.3cm}
\end{varblock}


\textbf{This will result to, assuming the 2 references information:}

\begin{varblock}[11cm]{\textbf{Output}}
\vspace{-0.3cm}
\begin{verbatim}
We appreciate the great work in [2], which let us 
write our work more beautifully[1, p. 200] .

References
[1] Leslie Lamport, \LaTeX: A Document Preparation System. Addison Wesley, 
Massachusetts, 2nd Edition, 1994.
[2] nuth, Donald Ervin. \The Art of Computer Programming (TAOCP)".
Retrieved February 01, 2014.
\end{verbatim}
\vspace{-0.3cm}
\end{varblock}


\end{frame}

\section{The natbib package}
\begin{frame}[t,fragile]{\textbf{\textbf{The natbib package}}}


The natbib package extends the citation in \LaTeX{} writing, catering to the requirements of most of the scientific literature. The basic commands are:\\
\vspace{0.2cm}

In the preamble (before the $\backslash$begin\{document\}) command we add:\\
\vspace{0.2cm}

\begin{enumerate}
\item \textcolor{red}{$\backslash$usepackage[options]\{natbib\}}. Options are usually- authoryear, square, comma, numbers etc. More will be defined later.\\
\vspace{0.2cm}

 \item \textcolor{red}{$\backslash$bibliographystyle\{styles\}}. Styles are provided by publishers e.g. rsc, nature, IEEEtranN, etc., but can also be customized. The file type is ``.bst''. 
 
\end{enumerate}

natbib also extends the \textcolor{red}{$\backslash$cite} command, we see how. 
\end{frame}


\begin{frame}[t,fragile]{\textbf{\textbf{The natbib package}}}


Several ways to cite using natbib package:\\
\vspace{0.3cm}


\small
\begin{tabular}{lcl}
\hline\rowcolor{gray!10}
\textbf{Citation Command} &  & \textbf{Output}  \\
\hline
 \rowcolor{almond}
$\backslash$citet\{goossens93\} & $\Rightarrow$ & Goossens et al. (1993)  \\

\rowcolor{ablue}
$\backslash$citep\{goossens93\} & $\Rightarrow$ & (Goossens et al., 1993)   \\

 \rowcolor{almond}
$\backslash$citet$\ast$\{goossens93\} & $\Rightarrow$ & Goossens, Mittlebach, and Samarin (1993) \\ 
 
\rowcolor{ablue}
$\backslash$citep$\ast$\{goossens93\} & $\Rightarrow$ & (Goossens, Mittlebach, and Samarin, 1993) \\ 

\rowcolor{almond}
$\backslash$citealt\{goossens93\} & $\Rightarrow$ & Goossens et al. 1993   \\

\rowcolor{ablue}
$\backslash$citealp\{goossens93\} & $\Rightarrow$ & Goossens et al., 1993   \\
 
\rowcolor{almond}
$\backslash$citet{ale91,rav92} & $\Rightarrow$ & Alex et al. (1991); Ravi et al. (1992)   \\ 

\rowcolor{ablue}
$\backslash$citep[see][chap.~4]{ale91} & $\Rightarrow$ & [see Alex 1991, chap. 4]  \\\hline
\end{tabular}\\
\vspace{0.3cm}

Finally, we re-visit the citation information input using natbib package.


\end{frame}

\begin{frame}[t,fragile]{\textbf{\textbf{The natbib package}}}



The natbib package provides an option for the $\backslash$bibitem command, i.e. we now use: \textcolor{red}{$\backslash$bibitem[custom text]\{ ref. identifier\}}. Examples:
\scriptsize
\begin{varblock}[10cm]{\textbf{Refernce details in input}}
\vspace{-0.3cm}
\begin{verbatim}
\begin{thebibliography}{99}
\bibitem[Leslie(1985)]{les85}Leslie Lamport, 1985. 
\LaTeX{} -A Document Preparation...
\bibitem[Donale(2000)]{don89}Donald E. Knuth, 1989. 
Typesetting Concrete Mathematics,.....
\end{thebibliography} 
\end{verbatim}
\vspace{-0.3cm}
\end{varblock}

\begin{varblock}[10cm]{\textbf{Output}}
\vspace{-0.3cm}
\begin{verbatim}
References

Leslie Lamport, 1985. LaTeX -A Document Preparation...
Donald E. Knuth, 1989. Typesetting Concrete Mathematics,...
\end{verbatim}
\vspace{-0.3cm}
\end{varblock}

\end{frame}

\section{Final Notes}

\begin{frame}{Final Notes}

\justifying

\LaTeX{} simplifies the referencing works of the document. The preceding slides only briefly introduced the referencing using \LaTeX. Following suggestions can be helpful.

\begin{enumerate}

\item Detail at the \href{http://en.wikibooks.org/wiki/LaTeX/Bibliography_Management}{wikipedia} site.\\
\vspace{0.2cm}

\item Different bibliography management software that can provide output for \LaTeX can be found \href{http://en.wikipedia.org/wiki/Comparison_of_reference_management_software}{here}\\
\vspace{0.2cm}

\item More packages to manage complex bibliography, e.g. \href{http://ctan.org/pkg/multibib}{multiple bibliography}, \href{http://ctan.org/pkg/chapterbib}{Chapter-wise bibliography}, \href{http://ctan.org/pkg/bibunits}{unit-wise bibliography} etc.

\item A good \LaTeX{} \href{http://www.sharelatex.com/learn/Main_Page}{tutorial}.

\end{enumerate} 


\end{frame}

{
\setbeamertemplate{footline}{} 
\begin{frame}{ \textbf{Next $\ldots$,}}
\pagestyle{plain}
\Huge

\textbf{Cross-referencing, Hyperlinking... \\ \vspace{0.5cm}Last but not least...,}


\end{frame}


}








